%!TeX root=../tese.tex
%("dica" para o editor de texto: este arquivo é parte de um documento maior)

\chapter{O painel administrativo}
\label{cap:implementacao}

\section{Arquitetura de software}
O painel adota arquitetura web modular e integra-se a serviços de autenticação,
cadastro de participantes, registros de viagens, contestações e processamento de
créditos/pagamentos. As principais conexões externas incluem o \textit{App Mobile}
dos participantes, o sistema de pagamentos (por exemplo, SPTrans), a equipe de
economistas, suporte/atendimento e coordenação.

\begin{figure}[H]
  \centering
  % Insira o arquivo em figuras/ com resolução adequada para impressão
  \IfFileExists{figuras/arquitetura.pdf}{%
    \includegraphics[width=.92\textwidth]{figuras/arquitetura.pdf}%
  }{%
    \IfFileExists{figuras/arquitetura.png}{%
      \includegraphics[width=.92\textwidth]{figuras/arquitetura.png}%
    }{%
      \fbox{\parbox{.9\textwidth}{Diagrama de arquitetura (placeholder). Inserir em figuras/arquitetura.pdf/png mostrando: Painel no centro; integrações com App Mobile, backend de dados, módulo de pagamentos (SPTrans), economistas, suporte e coordenação.}}%
    }%
  }%
  \caption{Arquitetura de alto nível: módulos e integrações do painel.}
  \label{fig:arquitetura}
\end{figure}

\section{Processo de desenvolvimento}
O desenvolvimento envolveu múltiplos stakeholders e ciclos curtos de validação:
economistas (por exemplo, Thainá), suporte/atendimento (Sávio), coordenação e
orientação (Fábio e Paulo), além da equipe técnica. As demandas foram
priorizadas e acompanhadas via registro de \textit{issues}, criação de
\textit{pull requests} e revisão colaborativa, com homologação contínua durante
o pré-teste, o período de inscrições, a preparação de julho e a execução do
piloto.

\section{Abas do painel e uso no piloto}
As figuras desta seção devem ser incluídas com resolução adequada para
impressão; além de legendas, descreva no texto o contexto e o que observar em
cada captura.

\subsection{Usuários}
Explicar fluxo de inscrição no LimeSurvey e script que buscava os usuários no
LimeSurvey e insere os usuários no banco de dados do app.

Após a inserção, o usuário pode ser visualizado na tela de usuários.
A tela apresenta uma tabela os dados principais do usuário, como id, nome, email,
cpf, bilhete único e se tem senha cadastrada.
O id do usuário era usado em processos importantes, como na gestão de bônus e coortes.
Nessa tabela, é possível buscar usuários por nome ou id, ordenar por coluna,
paginar e ajustar a densidade/quantidade de linhas por página.


\subsection{Coortes}
Explicar o que é uma coorte no contexto do BikeSP.
Coortes representam grupos experimentais/remuneração. Foi disponibilizada uma
página para criar e atualizar coortes e acompanhar um quadro sintético (ID,
remuneração e quantidade de membros). Além disso, também tem uma busca por participantes
das coortes, que inclui filtros por coorte e métricas de viagens, apoiando decisões
operacionais e o monitoramento do andamento do piloto.

É possível fazer uma atribuição em lote para configurar as coortes de múltiplos usuários
simultaneamente, explicar como essa função foi usada.

E também é possível exportar uma planilha com os usuários e suas coortes,
para facilitar tarefas administrativas recorrentes.


\subsection{Bônus}
Explicar o que é o bônus no contexto do BikeSP.
Gestão de bônus para usuários. Criação de bônus, ativação/desativação de bônus,
Atribuição de bônus aos participantes. 

A busca de bônus foi padronizada para apresentar datas e estados (ativo/inativo)
de forma consistente. Também passou a ser possível alterar o estado do bônus
diretamente na listagem, favorecendo a manutenção rápida das regras durante a
execução do piloto.


\subsection{Remuneração}
Explicar como era feito o processo de remuneração dos participantes.
O painel gera um CSV com os dados que deviam ser enviados à SPTrans para o pagamento
das viagens elegíveis.

O painel recebe um upload dos pagamentos realizados pela SPTrans, e debita os valores
do aplicativo do participante.


\subsection{Contestações}
Explicar o que são contestações e como podiam acontecer.
O painel permitia a visualização das contestações e o processo de aprovação/rejeição
das contestações. Nela, o administrador podia ver a viagem contestada, o motivo da
contestação, e decidir se a contestação era válida ou não.


\subsection{Viagens}
A tela de viagens exibia uma tabela com os dados principais da viagem, como id,
id do participante, data, local de início e fim, distância, remuneração e status.
A tabela apresentava busca por usuário, paginação e ajuste de densidade/quantidade 
de linhas por página. 

Ao clicar na viagem, o mapa contendo o trajeto da viagem era exibido.

Além disso, era possível exportar todas as viagens para CSV.

\subsection{Notificações}
Explicar o que são notificações e como podem ser enviadas.
Notificação por email e notificações push.
Podem ser direcionadas para um ou mais participantes, para participantes de uma coorte 
ou para todos os participantes.
As notificações também eram enviadas em cópia oculta para o email do administrador.

\subsection{Localizações}
Explicar o cadastro dos locais no LimeSurvey e como os locais deveriam ser usados
pelo usuário no app.
Gerenciamento de localizações do usuário, permitindo que ele alterasse suas localizações.

