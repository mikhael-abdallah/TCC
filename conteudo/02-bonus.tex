%!TeX root=../tese.tex
%("dica" para o editor de texto: este arquivo é parte de um documento maior)

% Conteúdo da subseção sobre Bônus
% Este arquivo é importado em 02-implementacao.tex

\textbf{Contexto e propósito dos bônus.} No contexto do piloto BikeSP, os bônus
representam uma ferramenta complementar de incentivo financeiro aos participantes,
independente das remunerações por quilômetro pedalado. Conforme planejado no desenho
do experimento, os bônus servem a múltiplos propósitos: (i) recompensar a conclusão
do curso gratuito ``Pedale com Segurança'' oferecido pelo Centro de Treinamento e
Educação de Trânsito (CETET) da CET; (ii) incentivar o preenchimento dos
questionários qualitativos aplicados em três momentos do piloto, garantindo maior
taxa de resposta e riqueza de dados; (iii) permitir ajustes administrativos e
correções pontuais de pagamento quando necessário; e (iv) viabilizar eventuais
campanhas promocionais temporárias para estimular a participação em momentos
estratégicos do experimento.

\textbf{Implementação técnica.} O painel administrativo oferece três
funcionalidades principais relacionadas a bônus: criação de novos tipos de bônus,
atribuição de bônus a usuários, e visualização e gerenciamento dos bônus
cadastrados. A arquitetura do sistema foi projetada para oferecer flexibilidade e controle sobre os incentivos.

\textbf{Criação de bônus.} Administradores podem cadastrar novos tipos de bônus
informando nome único (identificador), valor em reais, descrição textual, período
de validade (datas de início e fim, opcionais), e dois flags de controle: se o
bônus deve ser ativado automaticamente após criação, e se será visível aos
usuários no aplicativo móvel. A distinção entre bônus ativos/inativos e
visíveis/invisíveis oferece grande flexibilidade: bônus inativos não podem ser
concedidos (útil para pausar temporariamente um incentivo), enquanto bônus
invisíveis, mesmo quando concedidos e gerando remuneração, não aparecem na
listagem do aplicativo do usuário (útil para ajustes administrativos discretos ou
correções pontuais).

 \begin{figure}[H]
   \centering
   \includegraphics[width=0.95\textwidth]{figuras/criar_bonus.PNG}
   \caption{Formulário de criação de bônus no painel administrativo.}
   \label{fig:bonus_criar_form}
 \end{figure}

\textbf{Atribuição de bônus.} Para conceder um bônus a participantes, o
administrador informa os IDs dos usuários (separados por vírgula) e seleciona o
bônus desejado. O sistema valida diversas condições antes de processar a
concessão: o bônus deve estar ativo, não pode ter expirado (data de término não
ultrapassada), todos os usuários devem existir, e cada usuário pode receber cada
bônus apenas uma vez. Ao conceder um bônus com sucesso, o sistema não apenas registra a concessão, mas
também cria automaticamente uma remuneração correspondente ao valor do bônus e a
insere na fila de pagamentos (para posterior transferência aos cartões Bilhete
Único via Loja Virtual SPTrans). Caso algum usuário não atenda aos critérios, o
sistema reporta sucesso parcial, detalhando quantos usuários receberam o bônus e
quais apresentaram erros (por exemplo, ``5 sucesso(s), 2 erro(s): Usuário 101 já
possui este bônus; Usuário 203 não existe''). Esta funcionalidade de concessão em
lote reduz significativamente o tempo necessário para operações recorrentes, como
conceder o bônus do curso de segurança para todos os participantes que o
concluíram em determinada semana.

 \begin{figure}[H]
   \centering
   \includegraphics[width=0.95\textwidth]{figuras/atribuir_bonus.PNG}
   \caption{Formulário de atribuição de bônus aos participantes.}
   \label{fig:bonus_atribuir_form}
 \end{figure}

\textbf{Listagem e gerenciamento.} A tela de listagem exibe todos os bônus
cadastrados em formato tabular, apresentando ID, nome, valor, status de ativação,
visibilidade, descrição, período de validade e ações disponíveis. A interface
oferece busca por nome e paginação (5 itens por página). Cada linha da tabela
possui um botão para ativar ou desativar o bônus: bônus ativos exibem
botão vermelho ``Desativar'', enquanto bônus inativos exibem botão verde
``Ativar''. Importante notar que desativar um bônus não remove concessões já
realizadas nem afeta remunerações já geradas; apenas impede novas concessões
daquele tipo de bônus. Para facilitar a identificação dos IDs de usuários ao
atribuir bônus, a aba de bônus também oferece uma consulta integrada à tabela de
usuários, com busca por nome ou ID, ordenação por qualquer coluna e
paginação configurável da mesma forma que a tela de usuários.

\begin{figure}[H]
    \centering
    \includegraphics[width=0.95\textwidth]{figuras/bonus_listar.PNG}
    \caption{Listagem de bônus cadastrados com funcionalidades de busca, paginação e ativação/desativação.}
    \label{fig:bonus_listagem_form}
  \end{figure}
 

\textbf{Integração com o aplicativo móvel.} O aplicativo Android dos participantes
consome a API \texttt{POST /api/exibeBonus/} para listar os bônus concedidos ao
usuário autenticado. A API retorna apenas bônus marcados como visíveis, garantindo que ajustes administrativos invisíveis não
apareçam para os usuários finais. Esta separação entre visibilidade e concessão
oferece aos gestores do piloto a flexibilidade de realizar correções financeiras
sem necessariamente comunicá-las explicitamente aos participantes.

\textbf{Uso operacional no piloto.} Durante a execução do piloto, a
funcionalidade de bônus mostrou-se fundamental para diversos cenários operacionais:
concessão em lote do bônus do curso de segurança para centenas de participantes que
o concluíram simultaneamente; distribuição dos bônus dos questionários ao final de
cada período de coleta de dados qualitativos; correções pontuais quando erros de
geolocalização resultaram em viagens rejeitadas indevidamente (criando-se bônus
invisíveis de ajuste); A flexibilidade do sistema permitiu à
equipe operacional responder rapidamente a diferentes necessidades sem requerer
desenvolvimento de novas funcionalidades ou intervenções manuais no banco de dados.



