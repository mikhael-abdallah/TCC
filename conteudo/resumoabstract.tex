%!TeX root=../tese.tex
%("dica" para o editor de texto: este arquivo é parte de um documento maior)
% para saber mais: https://tex.stackexchange.com/q/78101

% As palavras-chave são obrigatórias, em português e em inglês, e devem ser
% definidas antes do resumo/abstract. Acrescente quantas forem necessárias.
\palavraschave{BikeSP, painel administrativo, incentivos financeiros, mobilidade urbana}

\keywords{BikeSP, administrative dashboard, financial incentives, urban mobility}

% O resumo é obrigatório, em português e inglês. Estes comandos também
% geram automaticamente a referência para o próprio documento, conforme
% as normas sugeridas da USP.
\resumo{
Este Trabalho de Conclusão de Curso apresenta o desenvolvimento de um painel
administrativo para o piloto do programa Bike~SP, iniciativa da Prefeitura de
São Paulo em parceria com a USP voltada à promoção do ciclismo urbano por meio
de incentivos financeiros. O trabalho descreve o contexto e implementação do painel,
que apoiou o pré-teste entre Abril e Maio, a incrição de 3 mil pessoas e o acompanhamento
do projeto piloto, que teve duração de quatro semanas.
O painel foi essencial para a gestão operacional do projeto, sendo importante para a 
gestão de cadastros, acompanhamento do uso do app, monitoramento das viagens 
e gestão dos pagamentos.
}

\abstract{
This Undergraduate Thesis presents the development of an administrative dashboard for
the Bike-SP program pilot, an initiative by the São Paulo City Hall in partnership with
USP to promote urban cycling through financial incentives. The work describes the context
and implementation of the dashboard, which supported a pre-test (April-May), the 
registration of 3,000 users, and the monitoring of the four-week pilot project. The 
dashboard was essential for the project's operational management, handling user 
registration, app usage tracking, trip monitoring, and payment management.
}
