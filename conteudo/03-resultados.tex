%!TeX root=../tese.tex
%("dica" para o editor de texto: este arquivo é parte de um documento maior)

\chapter{Resultados e avaliação}
\label{cap:resultados}

Este capítulo apresenta os resultados alcançados durante a execução do piloto BikeSP. A Seção~\ref{sec:resultados-atingidos} descreve os números e conquistas do experimento, incluindo volume de viagens, participação e descobertas relevantes. A Seção~\ref{sec:analise} apresenta uma análise de como o painel administrativo apoiou as perguntas de pesquisa e a execução do piloto.

\section{Resultados atingidos}
\label{sec:resultados-atingidos}
O piloto alcançou resultados expressivos em termos de coleta de dados e
engajamento dos participantes. Durante o pré-teste e o piloto, houve
\textbf{coleta intensiva de dados} via aplicativo e painel ao longo de cerca de
\textbf{dois meses} (julho e agosto). Os números do experimento foram:
\begin{itemize}
  \item \textbf{3~mil inscrições} recebidas no processo seletivo;
  \item \textbf{1217 participantes} selecionados para o experimento;
  \item \textbf{Mais de 29~mil viagens} registradas pelo aplicativo durante o
        experimento;
  \item \textbf{Mais de 150~mil quilômetros} pedalados no total.
\end{itemize}

Um resultado particularmente relevante foi a \textbf{retenção no uso do
aplicativo mesmo sem remuneração}. Entre as 10 viagens mais longas registradas,
9 foram realizadas fora dos períodos de remuneração (finais de semana),
indicando engajamento genuíno dos participantes com o sistema e com a prática do
ciclismo urbano, independentemente do incentivo financeiro imediato.

O projeto foi considerado um sucesso operacional. Em julho, o painel foi usado
para acompanhar o uso e prestar atendimento a usuários com dificuldades; em
agosto, executou-se o piloto principal em quatro semanas. A partir do segundo
teste com cerca de 40 participantes, foram mapeadas novas funcionalidades
críticas. Entre elas, destaca-se a \textbf{página do painel dedicada a corrigir
erros de geolocalização}, cuja operação reduziu consideravelmente as reclamações
dos usuários, permitindo que a equipe de suporte resolvesse problemas técnicos
de forma ágil e eficiente.

Entre os ganhos observados com o uso do painel:
\begin{itemize}
  \item \textbf{Eficiência}: atribuição de bônus em lote e ações em linha
        reduziram etapas repetitivas e tempo operacional.
  \item \textbf{Qualidade da informação}: padronização de datas/estados e
        visões dedicadas (Pessoas, Coortes) elevaram a confiabilidade das
        consultas e relatórios.
  \item \textbf{Governança}: listas de pagamento com trilhas de decisão e
        histórico de contestações facilitaram auditoria e reconciliação.
  \item \textbf{Suporte ao usuário}: ferramentas de correção de dados
        (especialmente geolocalização) reduziram reclamações e melhoraram a
        experiência dos participantes.
\end{itemize}


\section{Análise}
\label{sec:analise}
Como o painel apoia as perguntas de pesquisa e a execução do piloto\citep{interscity:pilotoBikeSP}.

O próximo capítulo conclui este trabalho, destacando as principais contribuições e discutindo possibilidades de trabalhos futuros e expansão do sistema.


