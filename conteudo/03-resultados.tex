%!TeX root=../tese.tex
%("dica" para o editor de texto: este arquivo é parte de um documento maior)

\chapter{Resultados e avaliação}
\label{cap:resultados}

\section{Resultados atingidos}
Durante o pré-teste e o piloto, houve \textbf{coleta intensiva de dados} via
aplicativo e painel. O projeto foi considerado um sucesso operacional; a equipe
de dados está analisando o conjunto coletado ao longo de cerca de \textbf{dois
meses} (julho e agosto). Em julho, o painel foi usado para acompanhar o uso e
prestar atendimento a usuários com dificuldades; em agosto, executou-se o
piloto principal em quatro semanas. Entre os ganhos observados:
\begin{itemize}
  \item \textbf{Eficiência}: atribuição de bônus em lote e ações em linha
        reduziram etapas repetitivas e tempo operacional.
  \item \textbf{Qualidade da informação}: padronização de datas/estados e
        visões dedicadas (Pessoas, Coortes) elevaram a confiabilidade das
        consultas e relatórios.
  \item \textbf{Governança}: listas de pagamento com trilhas de decisão e
        histórico de contestações facilitaram auditoria e reconciliação.
\end{itemize}


\section{Análise}
Como o painel apoia as perguntas de pesquisa e a execução do piloto\citep{interscity:pilotoBikeSP}.


