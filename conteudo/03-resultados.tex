%!TeX root=../tese.tex
%("dica" para o editor de texto: este arquivo é parte de um documento maior)

\chapter{Resultados e avaliação}
\label{cap:resultados}

Este capítulo apresenta os resultados alcançados durante a execução do piloto BikeSP. A Seção~\ref{sec:resultados-atingidos} descreve os números e conquistas do experimento, incluindo volume de viagens, participação e descobertas relevantes. A Seção~\ref{sec:avaliacao-painel} apresenta os resultados de um formulário de avaliação do painel administrativo aplicado aos principais usuários da ferramenta.

\section{Resultados atingidos}
\label{sec:resultados-atingidos}
O piloto alcançou resultados expressivos em termos de coleta de dados e
engajamento dos participantes. Durante o pré-teste e o piloto, houve
\textbf{coleta intensiva de dados} via aplicativo e painel ao longo de cerca de
\textbf{dois meses} (julho e agosto). Os números do experimento foram:
\begin{itemize}
  \item \textbf{3~mil inscrições} recebidas no processo seletivo;
  \item \textbf{1217 participantes} selecionados para o experimento;
  \item \textbf{Mais de 29~mil viagens} registradas pelo aplicativo durante o
        experimento;
  \item \textbf{Mais de 150~mil quilômetros} pedalados no total.
\end{itemize}

Um resultado particularmente relevante foi a \textbf{retenção no uso do
aplicativo mesmo sem remuneração}. Entre as 10 viagens mais longas registradas,
9 foram realizadas fora dos períodos de remuneração (finais de semana),
indicando engajamento genuíno dos participantes com o sistema e com a prática do
ciclismo urbano, independentemente do incentivo financeiro imediato.

O projeto foi considerado um sucesso operacional. Em julho, o painel foi usado
para acompanhar o uso e prestar atendimento a usuários com dificuldades; em
agosto, executou-se o piloto principal em quatro semanas. A partir do segundo
teste com cerca de 40 participantes, foram mapeadas novas funcionalidades
críticas. Entre elas, destaca-se a \textbf{página do painel dedicada a corrigir
erros de geolocalização}, cuja operação reduziu consideravelmente as reclamações
dos usuários, permitindo que a equipe de suporte resolvesse problemas técnicos
de forma ágil e eficiente.

Entre os ganhos observados com o uso do painel:
\begin{itemize}
  \item \textbf{Eficiência}: atribuição de bônus em lote e ações em linha
        reduziram etapas repetitivas e tempo operacional.
  \item \textbf{Qualidade da informação}: padronização de datas/estados e
        visões dedicadas (Pessoas, Coortes) elevaram a confiabilidade das
        consultas e relatórios.
  \item \textbf{Governança}: listas de pagamento com trilhas de decisão e
        histórico de contestações facilitaram auditoria e reconciliação.
  \item \textbf{Suporte ao usuário}: ferramentas de correção de dados
        (especialmente geolocalização) reduziram reclamações e melhoraram a
        experiência dos participantes.
\end{itemize}


\section{Avaliação do painel administrativo}
\label{sec:avaliacao-painel}

Ao final do piloto, foi aplicado um formulário de avaliação do painel administrativo aos principais usuários da ferramenta. O objetivo era coletar feedback sobre usabilidade, funcionalidades implementadas e identificar oportunidades de melhoria para futuras iterações do sistema. O formulário foi respondido por 5 usuários, sendo 3 que utilizaram o painel diariamente durante o piloto (Kaiky Cintra, Sávio Campos e Fabio Kon) e 2 usuários ocasionais (Paulo Meirelles e Nelson Lago).

\subsection{Utilidade e satisfação geral}

A avaliação revelou unanimidade quanto à utilidade do painel: todos os respondentes concordaram totalmente que o painel foi útil para a realização de suas tarefas e que facilitou o acesso e interpretação das informações necessárias. Este resultado é particularmente significativo considerando que o sistema foi desenvolvido especificamente para apoiar o piloto e precisava atender a diferentes perfis de usuários --- desde operações técnicas de suporte até acompanhamento estratégico por coordenadores.

Os usuários descreveram diversas tarefas críticas realizadas com o painel:

\begin{itemize}
  \item \textbf{Gestão de bônus em lote}: Concessão de bônus para 760 usuários simultaneamente, utilizando funcionalidade de inserção de IDs separados por vírgula, conforme descrito na Seção~\ref{sec:bonus}.
  \item \textbf{Correção de localizações}: Alteração intensiva de coordenadas GPS para localizações inválidas, utilizando a página de correção de geolocalização descrita na Seção~\ref{sec:localizacoes}.
  \item \textbf{Gerenciamento semanal de coortes}: Upload de arquivos CSV para atribuição de participantes às coortes a cada semana, processo fundamental para o desenho experimental descrito na Seção~\ref{sec:coortes}.
  \item \textbf{Monitoramento operacional}: Acompanhamento diário do total de viagens, valores pagos, e visualização de contestações para identificação proativa de problemas, descritos nas Seções~\ref{sec:viagens} e~\ref{sec:contestacoes}.
  \item \textbf{Análise geoespacial}: Visualização de mapas de viagens para verificar cobertura geográfica e validar trajetos em contestações.
\end{itemize}

Um comentário particularmente relevante foi: ``\textit{excelente trabalho. o painel viabilizou o piloto, sem ele seria impossível}'' (Fabio Kon), indicando que o sistema não apenas apoiou, mas foi essencial para a viabilidade operacional do experimento.

\subsection{Avaliações quantitativas}

O formulário incluiu avaliações quantitativas em escala de 1 a 5 para seis aspectos do painel. Os resultados foram:

\begin{itemize}
  \item \textbf{Facilidade de uso}: média 4,6 (60\% avaliaram como excelente)
  \item \textbf{Velocidade e desempenho}: média 4,6 (80\% avaliaram como excelente)
  \item \textbf{Clareza e organização}: média 4,4
  \item \textbf{Design e aparência}: média 4,2
  \item \textbf{Completude de funcionalidades}: média 3,8
  \item \textbf{Completude de informações}: média 3,6
\end{itemize}

Os pontos mais fortes identificados foram a facilidade de uso e a velocidade do sistema, ambos com média 4,6. A alta avaliação de usabilidade é notável considerando que os usuários não receberam treinamento formal. A velocidade do sistema foi destacada explicitamente: ``\textit{Velocidade excelente para todas as funcionalidades}'', aspecto crítico para operações que envolviam consultas diárias a tabelas com dezenas de milhares de registros.

A visualização de mapas das viagens foi citada por múltiplos respondentes como funcionalidade de destaque, tanto pela qualidade visual quanto pela utilidade prática: ``\textit{facilidade de identificar problemas de localização}'' e uso para ``\textit{verificar se havia pessoas de todas as regiões da cidade}''.

\subsection{Limitações e oportunidades de melhoria}

A completude de informações recebeu a menor média (3,6), refletindo limitações identificadas pelos usuários:

\begin{itemize}
  \item \textbf{Navegação entre entidades}: Falta de interligação direta entre as páginas de contestações, viagens e usuários. O fluxo atual requer anotar IDs manualmente e alternar entre páginas, processo que poderia ser otimizado com links diretos.
  \item \textbf{Busca limitada}: A página de usuários permite busca apenas por nome ou ID, sem possibilidade de buscar por CPF ou e-mail diretamente.
  \item \textbf{Clareza sobre campos}: Nem sempre é óbvio qual informação deve ser inserida em cada campo. Tooltips ou mensagens de ajuda poderiam esclarecer, por exemplo, se um campo espera o ID da viagem ou do usuário.
  \item \textbf{Visibilidade de regras}: Ausência de indicação sobre o motivo específico de rejeição de viagens, dificultando o diagnóstico de problemas reportados por participantes.
  \item \textbf{Rate limiting}: Limite de 50 atualizações por hora na correção de localizações, que foi atingido durante operações intensivas de correção.
  \item \textbf{Tabelas do banco}: O painel não expõe todas as 22 tabelas do banco de dados, faltando, por exemplo, consulta à tabela de pares válidos.
  \item \textbf{Compatibilidade}: Relatos de problemas de compatibilidade com o navegador Firefox, especialmente nas fases iniciais do projeto.
\end{itemize}

As sugestões de melhorias futuras incluíram: (i) página unificada por usuário, consolidando bilhetes, localizações e viagens; (ii) página unificada por localização, mostrando usuários e pares válidos; (iii) correção em lote para endereços com mesmo nome; (iv) melhor identificação da causa de rejeição de viagens (se foi pelo Google Activities ou pelo algoritmo de validação próprio); e (v) adição de tooltips e mensagens de ajuda para esclarecer informações esperadas em cada campo.

Importante notar que estas limitações foram identificadas após o uso do sistema e representam oportunidades de evolução, não impedimentos à operação do piloto. Como observado por um usuário: ``\textit{De modo geral a plataforma admin do bike SP apresenta boa interação com o usuário e boas funcionalidades}''.

\subsection{Síntese da avaliação}

A avaliação do painel administrativo confirma que o sistema atendeu aos objetivos estabelecidos na Seção~\ref{sec:objetivos-trabalho}, fornecendo: (i) visibilidade sobre cadastros e participantes; (ii) acompanhamento de viagens; (iii) cálculo e auditoria de créditos; (iv) tratamento de contestações; e (v) exportações para análise. A unanimidade na concordância sobre utilidade, combinada com as altas avaliações de usabilidade e desempenho, demonstra que o painel foi bem-sucedido como ferramenta central de operação e governança do piloto.

As limitações identificadas são típicas de sistemas desenvolvidos iterativamente em contexto de pesquisa, onde os requisitos emergem durante o uso. A documentação destas limitações e das sugestões de melhoria constitui contribuição valiosa para projetos similares e para a evolução futura do próprio sistema BikeSP.

O próximo capítulo conclui este trabalho, destacando as principais contribuições e discutindo possibilidades de trabalhos futuros e expansão do sistema.


