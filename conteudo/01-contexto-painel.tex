%!TeX root=../tese.tex
%("dica" para o editor de texto: este arquivo é parte de um documento maior)

\chapter{O Piloto BikeSP e seu painel administrativo}
\label{cap:contexto-painel}

Este capítulo apresenta o contexto do programa Bike~SP e do projeto piloto que foi desenvolvido para gerar evidências científicas para sua regulamentação. A Seção~\ref{sec:programa-bikesp} descreve o programa Bike~SP e sua origem na Lei Municipal 16.547/2016. A Seção~\ref{sec:piloto-bikesp} detalha o piloto experimental, incluindo cronograma, processo de inscrições e o desenho experimental com três coortes. Por fim, a Seção~\ref{sec:objetivos-trabalho} apresenta os objetivos específicos deste trabalho.

\section{Programa Bike~SP}
\label{sec:programa-bikesp}
O Bike~SP é um programa público que visa incentivar deslocamentos funcionais por
bicicleta por meio de créditos de mobilidade. Já o \textit{piloto} é uma etapa
experimental, delimitada no tempo e no escopo, concebida para testar variáveis
de implementação (por exemplo, níveis de incentivo), observar comportamentos de
uso e coletar evidências que subsidiem o aperfeiçoamento e o eventual
escalonamento do programa. Em outras palavras, o programa é a política pública;
o piloto é o experimento que informa sua execução
\citep{faria2023:bikespCaseStudy, interscity:pilotoBikeSP}.

Um projeto piloto é um experimento de pequena escala que permite testar a
viabilidade e efetividade de uma intervenção antes de escaloná-la para uma
população maior \citep{pearson2020}. A escolha do piloto como meio de guiar a
implementação do Bike~SP é motivada pela necessidade de dados precisos e do
mundo real.

Métodos tradicionais de pesquisa por questionários, embora úteis, frequentemente
falham em capturar a complexidade do comportamento real, pois pode haver
divergência significativa entre o que indivíduos afirmam que farão e o que
efetivamente fazem \citep{rundle2009}. Assim, o piloto simula a política real ao
compensar participantes por suas viagens de bicicleta, fornecendo dados sobre
como tal política poderia funcionar na prática.

Além disso, o programa Bike~SP tem potencial para integrar as agendas de justiça
social e ambiental, ao reduzir os custos de mobilidade de grupos de menor renda
e aumentar a equidade no transporte. Os créditos de transporte público obtidos
pedalando podem ser usados para pagar outras viagens (por ônibus, metrô ou trem),
beneficiando especialmente populações de baixa renda.

Dessa forma, o programa promove não apenas o aumento da participação modal do
ciclismo, mas também a equidade de mobilidade \citep{vanoutrive2020, jeekel2017}.

\section{Piloto do BikeSP}
\label{sec:piloto-bikesp}
O projeto piloto do BikeSP foi implementado entre abril e agosto, com a participação de 1.217 ciclistas selecionados entre aproximadamente 3.000 candidaturas. O projeto desdobrou-se em várias fases, desde um pré-teste inicial até a execução do experimento principal, conforme o cronograma detalhado a seguir.

\textbf{Cronograma.} \begin{itemize}
      \item \textbf{Abril--Maio}: Realização do pré-teste com participantes beta; o 
      painel administrativo foi utilizado desde o início para acompanhar registros e validar os fluxos do sistema.
      \item \textbf{Maio-Julho}: Período de inscrições, conduzido através da plataforma LimeSurvey.
      \item \textbf{Julho}: Ampliação do uso do aplicativo, fase em que a equipe de operações monitorou o uso e ofereceu suporte aos participantes através do painel.
      \item \textbf{Agosto}: execução do piloto principal, com duração de \textbf{4 
      semanas}.
    \end{itemize}

O formulário de inscrição criado na plataforma LimeSurvey permitiu
coleta padronizada de dados demográficos e de contexto de mobilidade. Com o
apoio de aparições na mídia, foram obtidas aproximadamente \textbf{3~mil
inscrições}. A seleção final considerou o desenho experimental e o balanceamento
de perfis e regiões, resultando em \textbf{1217 candidatos selecionados} para
participação no piloto. Para evitar viés, foram excluídos cicloativistas
reconhecidos, priorizando usuários representativos do público-alvo da política
pública.

O experimento foi estruturado com base em \textbf{três coortes}, que são grupos experimentais com diferentes níveis de
remuneração, conforme planejado pela equipe de economistas:
\begin{itemize}
  \item \textbf{Coorte 1 (controle)}: sem remuneração por viagens;
  \item \textbf{Coorte 2}: remuneração de R\$\,0,30 por quilômetro pedalado,
        limitado a 2 viagens por dia;
  \item \textbf{Coorte 3}: remuneração de R\$\,0,60 por quilômetro pedalado,
        também limitado a 2 viagens por dia.
\end{itemize}
A título de exemplo, um participante da Coorte~3 que percorresse 8~km em uma
viagem receberia R\$\,4,80 - valor próximo ao de uma passagem de ônibus na
época. Importante notar que \textbf{não houve remuneração nos finais de semana},
permitindo observar o comportamento de uso do aplicativo sem o incentivo
financeiro. Ao longo das quatro semanas do piloto, os participantes tiveram suas
coortes alternadas semanalmente, conforme desenho conduzido pelos economistas,
de modo a permitir a avaliação causal do efeito dos diferentes níveis de
incentivo.

% Depois ver se é necessário essa seção
%\textbf{Atores e processos.} Atores principais: participantes, equipe de
%operação e suporte, equipe de pesquisa, parceiros institucionais e equipe de
%desenvolvimento. Processos-chave: inscrição e triagem; amostragem; registro de
%viagens; validação de elegibilidade; cálculo de créditos; tratamento de
%\emph{contestações}; reporting e análise. Informações adicionais e materiais
%técnicos estão na wiki do aplicativo\citep{bikesp:wiki}.

\section{Objetivos do trabalho}
\label{sec:objetivos-trabalho}
Este trabalho teve como objetivo desenvolver um \textbf{painel administrativo} capaz
de dar suporte efetivo aos processos fundamentais do programa durante o piloto,
atuando como ferramenta central de operação e governança. Em alto nível, o
painel deve oferecer: (i) visibilidade sobre cadastros, elegibilidade e
amostras; (ii) acompanhamento de viagens registradas no aplicativo; (iii)
cálculo de créditos com trilhas de auditoria e apoio a pagamentos; (iv)
tratamento de contestações; e (v) exportações e insumos para monitoramento e
análise.

O próximo capítulo apresenta detalhadamente a implementação do painel administrativo, descrevendo a arquitetura de software do sistema BikeSP e cada uma das funcionalidades desenvolvidas para atender aos objetivos aqui descritos.


