%!TeX root=../tese.tex
%("dica" para o editor de texto: este arquivo é parte de um documento maior)

\chapter{Contexto do Bike~SP e painel administrativo}
\label{cap:contexto-painel}

\section{Programa Bike~SP}
O Bike~SP é um programa público que visa incentivar deslocamentos funcionais por
bicicleta por meio de créditos de mobilidade. Já o \textit{piloto} é uma etapa
experimental, delimitada no tempo e no escopo, concebida para testar variáveis
de implementação (por exemplo, níveis de incentivo), observar comportamentos de
uso e coletar evidências que subsidiem o aperfeiçoamento e o eventual
escalonamento do programa. Em outras palavras, o programa é a política pública;
o piloto é o experimento que informa sua execução
\citep{faria2023:bikespCaseStudy, interscity:pilotoBikeSP}.

\section{Piloto: cronograma, inscrições, seleção, atores e processos}
\textbf{Cronograma.} \begin{itemize}
  \item \textbf{Abril--Maio}: pré-teste com participantes beta; o painel foi
        utilizado desde o início para acompanhar registros e validar fluxos.
  \item \textbf{Três meses de inscrições}: processo conduzido via LimeSurvey;
        ao final, foram recebidas \textbf{cerca de 3000 candidaturas}.
  \item \textbf{Julho}: uso ampliado do aplicativo; a equipe operou o painel
        para monitorar o uso em tempo quase real e oferecer suporte a usuários
        com dificuldades.
  \item \textbf{Agosto}: execução do piloto principal em \textbf{4 semanas}.
\end{itemize}

\textbf{Inscrições e seleção.} O formulário de inscrição (LimeSurvey) permitiu
coleta padronizada de dados demográficos e de contexto de mobilidade. A seleção
considerou o desenho experimental e o balanceamento de perfis e regiões ao
atribuir participantes a grupos de controle e tratamento.

\textbf{Atores e processos.} Atores principais: participantes, equipe de
operação e suporte, equipe de pesquisa, parceiros institucionais e equipe de
desenvolvimento. Processos-chave: inscrição e triagem; amostragem; registro de
viagens; validação de elegibilidade; cálculo de créditos; tratamento de
\emph{contestações}; reporting e análise. Informações adicionais e materiais
técnicos estão na wiki do aplicativo\citep{bikesp:wiki}.

\section{Objetivos do trabalho}
Este TCC tem por objetivo desenvolver um \textbf{painel administrativo} capaz
de dar suporte efetivo aos processos fundamentais do programa durante o piloto,
atuando como ferramenta central de operação e governança. Em alto nível, o
painel deve oferecer: (i) visibilidade sobre cadastros, elegibilidade e
amostras; (ii) acompanhamento de viagens registradas no aplicativo; (iii)
cálculo de créditos com trilhas de auditoria e apoio a pagamentos; (iv)
tratamento de contestações; e (v) exportações e insumos para monitoramento e
análise.


