%!TeX root=../tese.tex
%("dica" para o editor de texto: este arquivo é parte de um documento maior)
% para saber mais: https://tex.stackexchange.com/q/78101

%% ------------------------------------------------------------------------- %%

% A introdução do TCC não deve ser numerada, mas deve aparecer no sumário;
% este modelo define o comando \chapter** para isso.
\chapter**{Introdução}
\label{cap:introducao-bikesp}

Este Trabalho de Conclusão de Curso documenta minha participação no
desenvolvimento de um painel administrativo para o programa Bike~SP, uma
iniciativa da Prefeitura de São Paulo em parceria com a USP que investiga o
uso de incentivos financeiros para promoção do ciclismo urbano. O projeto
prevê um piloto controlado com centenas de participantes, combinando técnicas
de ciência de dados e avaliação causal com uma perspectiva de justiça de dados
de mobilidade\citep{faria2023:bikespCaseStudy, interscity:pilotoBikeSP}.

O Bike~SP foi instituído por legislação municipal com o objetivo de conceder
créditos de mobilidade a cidadãos que realizam deslocamentos de trabalho ou
estudo utilizando bicicleta. Devido à escassez de precedentes de larga
escala e de evidências científicas em contextos semelhantes ao de São Paulo,
um estudo piloto vem sendo desenhado e conduzido por pesquisadores do IME-USP
e instituições parceiras, com documentação adicional no âmbito da disciplina
MAC0499\citep{analima:mac0499}.

O painel administrativo desenvolvido neste TCC tem como propósito apoiar a
operação do piloto e sua avaliação, oferecendo: (i) visibilidade sobre
cadastros, elegibilidade e amostras; (ii) acompanhamento de viagens registradas
no aplicativo móvel; (iii) cálculo e auditoria de créditos; e (iv) insumos para
monitoramento e análise. Ao longo do texto, são descritos contexto, requisitos, 
implementação do projeto e resultados alcançados.



