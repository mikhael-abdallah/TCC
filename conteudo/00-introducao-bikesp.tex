%!TeX root=../tese.tex
%("dica" para o editor de texto: este arquivo é parte de um documento maior)
% para saber mais: https://tex.stackexchange.com/q/78101

%% ------------------------------------------------------------------------- %%

% A introdução é agora um capítulo numerado
\chapter{Introdução}
\label{cap:introducao-bikesp}

A bicicleta é um meio de transporte sustentável que pode contribuir para a
melhoria da qualidade de vida urbana em diversas dimensões, como redução da
poluição ambiental, melhora da saúde pública e diminuição dos custos sociais e
individuais do transporte. Nesse sentido, políticas públicas que incentivam o
uso da bicicleta são cada vez mais importantes para o planejamento de transporte
urbano \citep{sheller2018, jeekel2017}. Em São Paulo, maior cidade do Brasil com
mais de 12 milhões de habitantes e 24 milhões de viagens diárias, apenas 0,9\%
de todas as viagens eram feitas por bicicleta em 2017 \citep{metro2017}.
Recentemente, a Prefeitura estabeleceu a meta de elevar a participação modal do
ciclismo para 3,2\% até 2028 \citep{saopauloCityHall2022}.

O programa Bike~SP tem origem na \textbf{Lei Municipal 16.547/2016}
\citep{municipalLegislation2016}, que prevê a concessão de créditos de
mobilidade a cidadãos que realizam deslocamentos utilizando bicicleta. A lei
especifica que esses créditos serão dados aos usuários registrados no sistema
\emph{Bilhete Único}, podendo ser utilizados para pagamento de ônibus, metrô e
trem. No entanto, a implementação foi adiada devido à ausência de decreto
regulamentar que especifique aspectos práticos da política, como critérios de
elegibilidade, valor do benefício e método de medição das distâncias percorridas.

Incentivos financeiros para ciclistas já existem em alguns países, principalmente
na Europa, como na Holanda (subsídio de \texteuro{}0,25 por quilômetro) e na
Bélgica \citep{tennant2022, brusselsTimes2022}. Entretanto, devido à escassez de
implementações anteriores em contextos semelhantes ao de São Paulo e de
evidências científicas que orientem seu desenho e execução, decidiu-se fornecer
suporte científico à política pública através da implementação de um projeto
piloto \citep{faria2023:bikespCaseStudy, pearson2020}.

A partir de 2023, o IME-USP iniciou o projeto de pesquisa em parceria com o
grupo INCT InterSCity, com o objetivo de elaborar um experimento controlado
capaz de gerar evidências empíricas para guiar a regulamentação efetiva desta
lei. Este estudo piloto foi desenhado e conduzido por pesquisadores do IME-USP
e instituições parceiras, incluindo especialistas em mobilidade cicloviária e a
\emph{Eindhoven University of Technology} (TU/e).

O presente trabalho, desenvolvido em 2025, buscou viabilizar a execução do
experimento através de uma equipe de desenvolvimento e
suporte composta por 6 pessoas do grupo de pesquisa, projetando e implementando
uma versão do sistema com funcionalidades estendidas para atender às
demandas operacionais e de pesquisa do piloto.

O painel administrativo desenvolvido neste trabalho tem como propósito apoiar a
operação do piloto e sua avaliação, oferecendo: (i) visibilidade sobre
cadastros, elegibilidade e amostras; (ii) acompanhamento de viagens registradas
no aplicativo móvel; (iii) cálculo e auditoria de créditos; e (iv) insumos para
monitoramento e análise de dados.

Este trabalho está organizado da seguinte forma: o Capítulo~\ref{cap:contexto-painel} apresenta o contexto do Piloto BikeSP, descrevendo o programa Bike~SP, o desenho experimental do piloto com suas coortes e cronograma, e os objetivos do trabalho; o Capítulo~\ref{cap:implementacao} detalha a implementação do painel administrativo, incluindo a arquitetura de software e as funcionalidades desenvolvidas para cada aba do sistema; o Capítulo~\ref{cap:resultados} apresenta os resultados alcançados durante a execução do piloto e uma análise dos impactos do painel nas operações; por fim, o Capítulo~\ref{cap:conclusao} conclui o trabalho e discute trabalhos futuros.



