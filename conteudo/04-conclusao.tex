%!TeX root=../Monografia - Mikhael Pinto.tex
%("dica" para o editor de texto: este arquivo é parte de um documento maior)

\chapter{Conclusão}
\label{cap:conclusao}

Este trabalho apresentou o contexto do programa Bike~SP e a implementação de um
painel administrativo para apoiar o piloto. As principais contribuições foram a
estruturação de módulos voltados à operação e análise de dados, alinhados ao
desenho experimental documentado \citep{faria2023:bikespCaseStudy}.

O sistema desenvolvido permitiu a coleta bem-sucedida de dados de mais de 30~mil
viagens e o gerenciamento operacional de 1217 participantes ao longo de quatro
semanas de experimento. Os dados coletados através do software desenvolvido
serão essenciais para a elaboração de um \textbf{documento de recomendações para
a regulamentação da Lei Municipal 16.547/2016} em São Paulo, fornecendo
evidências empíricas sobre a efetividade de diferentes níveis de incentivo
financeiro na promoção do ciclismo urbano.

Os resultados positivos do piloto abrem possibilidades para trabalhos futuros,
incluindo: (i) adaptação do sistema para suportar um número maior de
participantes na implantação em larga escala do programa em São Paulo; (ii)
replicabilidade do sistema em outras cidades --- o sistema já está sendo
utilizado em dois pilotos diferentes na cidade de Fortaleza, o que já demanda adaptações para generalizar o código,
 aumentando a sua flexibilidade e capacidade de replicação em diferentes contextos urbanos; (iii) estudos aprofundados 
sobre os padrões de mobilidade identificados, perfis de usuários e efetividade 
comparativa dos diferentes níveis de incentivo; e (iv) refinamento contínuo das 
funcionalidades do painel com base no feedback da equipe operacional e dos pesquisadores.


Do ponto de vista do aprendizado pessoal, este trabalho proporcionou experiências 
valiosas em múltiplas dimensões. O processo de desenvolvimento colaborativo com 
uma equipe multidisciplinar --- incluindo economistas, pesquisadores de mobilidade, 
profissionais de suporte e desenvolvedores --- demonstrou a importância da 
comunicação clara e da compreensão de diferentes perspectivas para o sucesso do 
projeto. A necessidade de adaptar requisitos dinamicamente durante o piloto, 
respondendo a feedbacks em tempo real da equipe operacional, ensinou a importância 
da flexibilidade e do desenvolvimento iterativo em projetos de pesquisa.

A experiência de ver o sistema sendo utilizado efetivamente em operações reais, 
com mais de 1200 participantes gerando dezenas de milhares de registros, 
proporcionou compreensão prática sobre escalabilidade, desempenho e usabilidade 
que dificilmente seria alcançada em um contexto puramente acadêmico. Além disso, 
a replicação do sistema em outras cidades evidenciou a importância de decisões 
arquiteturais que favoreçam modularidade e generalização, permitindo que o 
código seja adaptado a diferentes contextos urbanos e desenhos experimentais.

Por fim, o trabalho reforçou o papel essencial que sistemas bem 
projetados desempenham na viabilização de políticas públicas baseadas em 
evidências, demonstrando que a tecnologia pode ser uma ponte efetiva entre 
pesquisa científica e implementação prática de soluções para problemas urbanos 
complexos.


