%!TeX root=../tese.tex
%("dica" para o editor de texto: este arquivo é parte de um documento maior)

\chapter{Conclusão}
\label{cap:conclusao}

Este trabalho apresentou o contexto do programa Bike~SP e a implementação de um
painel administrativo para apoiar o piloto. As principais contribuições foram a
estruturação de módulos voltados à operação e análise de dados, alinhados ao
desenho experimental documentado \citep{faria2023:bikespCaseStudy}.

O sistema desenvolvido permitiu a coleta bem-sucedida de dados de mais de 29~mil
viagens e o gerenciamento operacional de 1217 participantes ao longo de quatro
semanas de experimento. Os dados coletados através do software desenvolvido
serão essenciais para a elaboração de um \textbf{documento de recomendações para
a regulamentação da Lei Municipal 16.547/2016} em São Paulo, fornecendo
evidências empíricas sobre a efetividade de diferentes níveis de incentivo
financeiro na promoção do ciclismo urbano.

Os resultados positivos do piloto abrem possibilidades para trabalhos futuros,
incluindo: (i) adaptação do sistema para suportar um número maior de
participantes na implantação em larga escala do programa em São Paulo; (ii)
replicabilidade do sistema em outras cidades --- o sistema está sendo
considerado para adoção pela Prefeitura de Fortaleza, o que demandará adaptações
e generalizações do código; (iii) estudos aprofundados sobre os padrões de
mobilidade identificados, perfis de usuários e efetividade comparativa dos
diferentes níveis de incentivo; e (iv) refinamento contínuo das funcionalidades
do painel com base no feedback da equipe operacional e dos pesquisadores.

A experiência adquirida no desenvolvimento e operação deste painel demonstra o
papel fundamental de ferramentas administrativas bem projetadas na viabilização
de experimentos científicos de políticas públicas de mobilidade urbana.


